\documentclass{../Vorlage/sebDenCls}
\usepackage{graphicx}
\usepackage{caption}
\usepackage{subcaption}
\usepackage{listings}
\usepackage{todo}
\usepackage{longtable}
\lstset{language=C++,basicstyle=\footnotesize}
\graphicspath{ {screenshots_documentation/} }

\setcounter{section}{3}

\begin{document}
\fach{Elektronische Bildverarbeitung}
\nr{2}
\abgabe{13.06.2017}
\semester{SoSe17}
\blatt{Udo Frese}{Lukas Bertram}{Sebastian Bliefert}{Johannes Hochbein}{Pascal Pieper}


\section{}
Unsere Quest ist die automatische Erkennung von Nummernschildern. Dieses System soll genutzt werden um z.B. eine Schranke oder ein Tor zu öffnen. Für uns gibt es hierbei eine tatsächliche Anwendung, da der Vater eines Gruppenmitgliedes das Tor zu seinen Betrieb über dieses System öffnen lassen möchte. 

Der besondere Vorteil des Systems ist, dass erwartete Kunden, Lieferanten und Partner (bei bekannten Kennzeichen) automatisch auf das Gelände gelangen, da Nummernschilder einfach ein und ausgetragen werden können. Des weiteren wäre es Möglich sich benachrichtigen zu lassen wenn Kennzeichen „XY“ angekommen ist. So kann der jeweils verantwortliche Mitarbeiter seinen „Besuch“ in Empfang nehmen oder ähnliches. 

In unseren Speziellen Fall soll die Kamera an einem Tor angebracht werden. Das Tor kann bereits Automatisch geöffnet werden. Unser System muss also „nur“ ein entsprechendes Signal an den vorhandenen Mechanismus senden. Eine automatische Benachrichtigung wird derzeit nicht benötigt. 

Die Position von Kennzeichen variiert zwischen den verschiedenen Fahrzeugklassen zwar etwas die Unterschiede sind aber nicht riesig, sodass eine Kamera mit „Normalen“ objektiv (Weitwinkel nicht benötigt) auf Kniehöhe für alle zu erwartenden Fälle ausreichend ist. 

Die Anordnung hat den Vorteil, dass ankommende Fahrzeuge bereits in einer gewissen Distanz gesehen werden, so kann das Tor bereits geöffnet werden und es ergeben sich keine Wartezeiten. Sollten Schwierigkeiten bei der Erkennung entstehen würde das Fahrzeug immer näher kommen und das Kennzeichen immer besser zu erkennen sein, sodass es bei Schwierigkeiten eventuell erst etwas später zu einer Erkennung kommt dies aber hauptsächlich einen Einfluss auf den „Komfort“ des Systems hat und nicht zu einen tatsächlichen Ausfall führt.

Da Fahrzeuge auch bei Dunkelheit auf das Gelände gelangen sollen bietet sich der Einsatz von Infrarot an. Denn Kennzeichen sind dafür gemacht um Infrarot-Strahlen zu reflektieren. Die Kombination aus einem Infrarotstrahler in LED -ausführung und einen Filter vor der Kamera kann überflüssige Bildinhalte „Ausblenden“ und bei Dunkelheit(sichtbarer Strahlenraum) Kennzeichen entsprechend belichten.

Hieraus ergeben sich folgende Anforderungen:
\begin{itemize}
\item Zu erkennende Kennzeichen: europäische Standardkennzeichen
\item Montage der Kamera in Grundstückszufahrten
\item Abstand Fahrzeug zur Kamera 1-2m
\item Montage auf einer Höhe von 0,5-1m
\item Leichte Winkeländerungen des Fahrzeugs dürfen nicht zu Fehlerkennungen führen
\item Korrekte Funktion zu allen Tageszeiten
\end{itemize}	


\section{}

\section{}
Der Hardwareaufbau unseres Lösungsansatzes ist grundlegend bereits durch die Aufgabenstellung definiert.\\
Die Kamera wird in, in der Aufgabenstellung definierter Höhe, zum Beispiel an einem Tor angebracht. Dabei ist aus Datenschutzgründen sie leicht nach unten gerichtet, sodass Fußgänger deutlich unterhalb des Kopfes abgeschnitten werden. Ihr Schärfebereich wird auf ein bis zwei Meter abstand eingestellt. Beleuchtet wird mit einem Koaxialen Ringlicht.\\
\\
Aus Gründen der Energiesparsamkeit und des Datneschutzes verwenden wir einen Trigger (zum Beispiel einen Ultraschalldistanzsensor), der im Distanzbereich von ein bis zwei Meter die Bilderkennung auslöst.\\
\\
\textbf{Algorithmen:}\\\\
\begin{longtable}{lllp{4cm}}
\textbf{Methodenname} & \textbf{Eingabe} & \textbf{Ausgabe} & \textbf{Beschreibung}\\
\hline
cropImage & ein Bild & ein Bild & Die Seiten des Bildes werden abgeschnitten, da hier kein relevantes Material vorliegt.\\\hline
preprocess & ein Bild & ein Bild & Helligkeit und Kontrast werden so angepasst, dass die Nummernschilder gut erkennbar sind\\\hline
detectLines & ein Bild & mehrere Linien & Linien im Bild werden mittels geeignetem Liniendetektor (z.B. Hough-Linedetector) erkannt und ausgegeben\\\hline
findPlates & mehrere Linien & mehrere Trapeze & Erkennen von Trapezen (also von potentiellen Nummernschildern)\\\hline
cropPlate & ein Trapez, ein Bild & ein Bild & Schneide das eingegebene Bild so zu, dass das übergeben Trapez knapp enthalten ist.\\\hline
dewarp & einTrapez, ein Bild & ein Bild & Entzerre das Nummernschild anhand des übergebenen Trapezes\\\hline
getText & ein Bild & ein String & Finde Text im übergebenen Bild\\\hline
lookupPlate & ein String & ein Integer & Nachschlagen des Nummernschildes in einer Datenbank o.Ä. und rückgabe der zugehörigen Eintrags-ID\\\hline
act & ein Integer & nichts & Auslösen der hinterlegten Aktion wie zum Beispiel das öffnen eines Tores\\

\end{longtable}
\end{document}


