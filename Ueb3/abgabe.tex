\documentclass{../Vorlage/sebDenCls}
\usepackage{graphicx}
\usepackage{caption}
\usepackage{subcaption}
\usepackage{listings}
\usepackage{todo}
\usepackage{longtable}
\lstset{language=C++,basicstyle=\footnotesize}
\graphicspath{ {Bilder/Doku/} }

\setcounter{section}{3}

\begin{document}
\fach{Echtzeitbildverarbeitung}
\nr{3}
\abgabe{13.06.2017}
\semester{SoSe17}
\blatt{Udo Frese}{Lukas Bertram}{(lbertram@uni-bremen.de)}{Sebastian Bliefert}{(bliefert@uni-bremen.de)}{Johannes Hochbein}{(hochbein@uni-bremen.de)}{Pascal Pieper}{(ppieper@uni-bremen.de)}


\section{Der Algorithmus}
Erklärung des Frameworks, Abfolge der GROBEN Algos
\#P
\subsection{CropImage}
muss nicht, weil. \$J
\subsection{Preprocessing}
Dunkel abschneiden.
\#L

\subsection{DetectLines}
Wir holen keine Linien, weil das zu zu vielen Fehlerkennungen geführt hat.
\#L
\subsection{findPlates}
Canny, contour, convex, nach Fläche sortieren, approx poly vereinfachen.
\#L
\subsection{cropPlate}
Crops.
\$B
\subsection{deWarp}
Warps. Wurst.
\$B
\subsection{getText}
Findet Text mit tesseract.
\#P
\subsection{lookupPlate}
Sucht nach definierten Kennzeichen in erkanntem Text.
\#P
\section{Ergebnisse}

Bilder von funktionierenden debug Ausgaben. Sagen, dass sowohl img als auch live geht.
\#J
\section{Ausblick}

Was geht nicht? Die Schriftart ist schwer für tesseract. Anlernen müsste man.
Live geht es besser, weil wir mehr Versuche haben, und dann einzelne Buchstaben 
\#J
\end{document}


