\documentclass{../Vorlage/mat}

\begin{document}
\maketitle{Sebastian Bliefert, Lukas Bertram}{}{Pascal Pieper, Johannes Hochbein}{}{}{}{XX.XX.2017}{1} \\

\section*{Aufgabe 1}
Ja.

TODO: 
-HSV wandlung
-thresholds für Farben finden
-Konzept aufschreiben
-Textaufgabe vom Schluss
-Optimierung (eher egal)

\section*{Aufgabe 3}

Konzept:
Der Keratograf sendet ein definiertes Muster aus konzentrischen, abwechselnd schwarzen und weißen Kreisen in Richtung Auge. Je nach Position des Auges vor dem Keratografen ergibt sich eine entsprechende Spiegelung der Ringe auf der Oberfläche der Hornhaut. Die Kamera ist im Zentrum des Keratografen platziert und nimmt das Bild der Spiegelung auf. Nach Bestimmung des gemeinsamen Mittelpunkts der Kreise wird der Abstand der Übergänge von Schwarz zu Weiß und von Weiß zu Schwarz entlang eines radialen Schnitts vermessen. Dazu wird eine Halbgerade festgelegt, die zunächst horizontal durch den Mittelpunkt verläuft und in diesem beginnt. Mit definierter Schrittweite wird der Schnittpunkt der Geraden mit einem fiktiven, zum Mittelpunkt konzentrischen Kreis auf diesem Kreis umlaufend verschoben. In jedem Zyklus werden nach einem "Weiterdrehen" alle Kanten in zur Halbgeraden senkrechter Richtung zunächst gefunden und dann in ihrem Abstand zum Mittelpunkt (Beginn der Halbgeraden und damit Urpsrung des zur Vermessung genutzten Koordinatensystems) vermessen. 

Bewertung der Messwerte:
Der theoretisch perfekte Abstand zwischen den Kanten ist bekannt für eine perfekt kugelfömige Honrhaut. Die Differenzen zwischen den theoretischen und den tatsächlich gemessenen Abständen gibt Aufschluss über die Form der Hornhaut: 
- Ist der gemessene Abstand kleiner als der theoretisch erwartete Wert, so ist die Hornhaut an dieser Stelle konkaver als die perfekte Kugel.
- Ist größer als der theoretisch erwartete Wert, so ist die Hornhaut an dieser Stelle konvexer als die perfekte Kugel.

\end{document}
